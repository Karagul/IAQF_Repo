%% LyX 2.2.3 created this file.  For more info, see http://www.lyx.org/.
%% Do not edit unless you really know what you are doing.
\documentclass[english]{IEEEtran}
\usepackage[T1]{fontenc}
\usepackage[latin9]{luainputenc}
\usepackage{array}

\makeatletter

%%%%%%%%%%%%%%%%%%%%%%%%%%%%%% LyX specific LaTeX commands.
%% Because html converters don't know tabularnewline
\providecommand{\tabularnewline}{\\}

\makeatother

\usepackage{babel}
\begin{document}

\title{Research Challenge on Prediction of Credit Spread by Machine Learning}

\author{Hui Cai, Israel Diego, Yifei Lu, Rick (Yuankang) Xiong, Xinye Xu}
\maketitle

\section*{Abstract}

Super-resolution microscopy has become essential for the study of
nanoscale biological processes. This type of imaging often requires
the use of specialised image analysis tools to process a large volume
of recorded data and extract quantitative information. In recent years,
our team has built an open-source image analysis framework for super-resolution
microscopy designed to combine high performance and ease of use.

\section*{Introduction}

\begin{tabular}{|l||l||>{\raggedright}p{0.15\paperwidth}|}
\hline 
Abbreviation & Name of Variable & Variable Description\tabularnewline
\hline 
\hline 
CS & Credit Spread & Credit spread of US investment grade corporate bonds (e.g. BofAML
US Corporate AAA) over US Treasuries (e.g. 10-Year Treasury Constant
Maturity Rate). \tabularnewline
\hline 
\hline 
LEVER & Leverage & Long-term debt divided by total assets\tabularnewline
\hline 
\hline 
R & Risk-free rate & Yield of 10-Year Treasury bonds.\tabularnewline
\hline 
\hline 
SLOPE & Slope of terms & Yield on 10-year Treasury bonds minus yield on 2-year Treasury bonds,\tabularnewline
\hline 
\hline 
VIX & Volatility Index & Average implied volatility of eight near-the-money options on the
S\&P 100 index?\tabularnewline
\hline 
\hline 
RETSP & S\&P 500 index return & Return on the S\&P 500 stock index (frequency?)\tabularnewline
\hline 
\hline 
JUMP & Change in slope of Volatility Smirk & A large size jump on S\&P 100 index, calculated using out-of-the money
puts as well as at- and in-the-money call options?\tabularnewline
\hline 
\end{tabular}

\section*{Methodologies}

\section*{Bibliography}
\end{document}
